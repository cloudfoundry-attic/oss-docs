\input{mmd-article-header}
\def\mytitle{BOSH Tutorial}
\def\myauthor{VMware 2012 - Cloud Foundry - Internal Use Only}
\def\latexmode{memoir}
\input{mmd-article-begin-doc}
\chapter{Introduction}
\label{introduction}

This tutorial will guide you through the process of deploying a multi-tier WordPress installation using BOSH. Due to its simplicity, WordPress is a good way to learn BOSH, but it is not a realistic use case. Most, if not all, of BOSH’s utility will be in the context of deploying Cloud Foundry.

\chapter{Prerequisites}
\label{prerequisites}

\begin{itemize}
\item A running development environment with a BOSH Director and an uploaded stemcell.

\item Access to the BOSH Gerrit server

\item A clean Ubuntu 10.04 LTS running in VMWare Fusion

\end{itemize}

\chapter{Installing BOSH on an Ubuntu VM}
\label{installingboshonanubuntuvm}

\section{Install Ruby via rbenv}
\label{installrubyviarbenv}

\begin{enumerate}
\item Bosh is written in Ruby. Let's install Ruby's dependencies

\begin{verbatim}
sudo apt-get install git-core build-essential libsqlite3-dev curl libmysqlclient-dev libxml2-dev libxslt-dev libpq-dev
\end{verbatim}


\item Get the latest version of rbenv

\begin{verbatim}
cd
git clone git://github.com/sstephenson/rbenv.git .rbenv
\end{verbatim}


\item Add \texttt{\ensuremath{\sim}\slash .rbenv\slash bin} to your \texttt{\$PATH} for access to the \texttt{rbenv} command-line utility

\begin{verbatim}
echo 'export PATH="$HOME/.rbenv/bin:$PATH"' >> ~/.bash_profile
\end{verbatim}


\item Add rbenv init to your shell to enable shims and autocompletion

\begin{verbatim}
echo 'eval "$(rbenv init -)"' >> ~/.bash_profile
\end{verbatim}


\item Download Ruby 1.9.2

\begin{verbatim}
wget http://ftp.ruby-lang.org/pub/ruby/1.9/ruby-1.9.2-p290.tar.gz
\end{verbatim}


\item Unpack and install Ruby

\begin{verbatim}
 tar xvfz ruby-1.9.2-p290.tar.gz
 cd ruby-1.9.2-p290
 ./configure --prefix=$HOME/.rbenv/versions/1.9.2-p290
 make
 make install
\end{verbatim}


\item Restart your shell so the path changes take effect

\begin{verbatim}
source ~/.bash_profile
\end{verbatim}


\item Set your default Ruby to be version 1.9.2

\begin{verbatim}
rbenv global 1.9.2-p290
\end{verbatim}


\end{enumerate}

\section{Install Local BOSH and BOSH Releases}
\label{installlocalboshandboshreleases}

\begin{enumerate}
\item Sign up for the Cloud Foundry Gerrit server at \href{http://cloudfoundry-codereview.qa.mozycloud.com/gerrit}{http:/\slash cloudfoundry-codereview.qa.mozycloud.com\slash gerrit}\footnote{\href{http://cloudfoundry-codereview.qa.mozycloud.com/gerrit}{http:/\slash cloudfoundry-codereview.qa.mozycloud.com\slash gerrit}}

\end{enumerate}

\textbf{NOTE: PUBLIC GERRIT IN FINAL DRAFT}

\begin{enumerate}
\item Set up your ssh public key (accept all defaults)

\begin{verbatim}
ssh-keygen -t rsa
\end{verbatim}


\item Copy your key from \texttt{\ensuremath{\sim}\slash .ssh\slash id\_rsa.pub} into your Gerrit account

\end{enumerate}

1.Create \ensuremath{\sim}\slash .gitconfig as follows (Make sure that the email specified is registered with gerrit):

\begin{adjustwidth}{2.5em}{2.5em}
\begin{verbatim}

    [user]
    name = YOUR_NAME
    email = YOUR_EMAIL
    [alias]
    gerrit-clone = !bash -c 'gerrit-clone $@' -

\end{verbatim}
\end{adjustwidth}

\begin{enumerate}
\item Clone gerrit tools using git

\begin{verbatim}
git clone git@github.com:vmware-ac/tools.git
\end{verbatim}


\end{enumerate}

\textbf{NOTE: PUBLIC TOOLS REPO IN FINAL DRAFT}

\begin{enumerate}
\item Add gerrit-clone to your path

\begin{verbatim}
echo 'export PATH="$HOME/tools/gerrit/:$PATH"' >> ~/.bash_profile
\end{verbatim}


\item Restart your shell so the path changes take effect

\begin{verbatim}
source ~/.bash_profile
\end{verbatim}


\item Clone BOSH repositories from Gerrit

\begin{verbatim}
git gerrit-clone ssh://cloudfoundry-codereview.qa.mozycloud.com:29418/bosh-sample-release
git gerrit-clone ssh://cloudfoundry-codereview.qa.mozycloud.com:29418/release.git
git gerrit-clone ssh://cloudfoundry-codereview.qa.mozycloud.com:29418/bos.git
\end{verbatim}


\item Run some rake tasks to install the BOSH CLI

\begin{verbatim}
cd ~/bosh
rake bundle_install
cd cli
bundle exec rake build
gem install pkg/bosh_cli-x.x.x.gem
\end{verbatim}


\end{enumerate}

\section{Deploy to your BOSH Environment}
\label{deploytoyourboshenvironment}

With a fully configured environment, we can begin deploying the sample application to our environment. As listed in the prerequisites, you should already have an environment running, as well as the IP address of the BOSH Director. Ask your BOSH technical contact for help if you need it.

\subsection{Point BOSH at a Target and Clean your Environment}
\label{pointboshatatargetandcleanyourenvironment}

\begin{enumerate}
\item Target your director (this IP is an example) \textbf{NOTE: EXAMPLE WORKS FOR INTERNAL USE (u: admin \slash  p: admin)}

\begin{verbatim}
bosh target 172.23.128.219:25555 
\end{verbatim}


\item Check the state of your BOSH settings.

\begin{verbatim}
bosh status
\end{verbatim}


\item The result of your status will be akin to:

\begin{verbatim}
Target         dev48 (http://172.23.128.219:25555) Ver: 0.3.12 (01169817)
UUID           4a8a029c-f0ae-49a2-b016-c8f47aa1ac85
User           admin
Deployment     not set
\end{verbatim}


\item List previous deployments (we will remove them in a moment)

\begin{verbatim}
bosh deployments
\end{verbatim}


\item The result of \texttt{bosh deployments} should be akin to:

\begin{verbatim}
+-------+
| Name  |
+-------+
| dev48 |
+-------+
\end{verbatim}


\item Delete the existing deployments (ex: dev48) 

\begin{verbatim}
bosh delete deployment dev48
\end{verbatim}


\item Answer \texttt{yes} to the prompt and wait for the deletion to complete

\item List previous releases (we will remove them in a moment)

\begin{verbatim}
`bosh releases`
\end{verbatim}


\item The result of \texttt{bosh releases} should be akin to:

\begin{verbatim}
+----------+------------+
| Name     | Versions   |
+----------+------------+
| appcloud | 47, 55, 58 |
+----------+------------+
\end{verbatim}


\item Delete the existing releases (ex: appcloud) 

\begin{verbatim}
bosh delete release appcloud
\end{verbatim}


\item Answer \texttt{yes} to the prompt and wait for the deletion to complete

\end{enumerate}

\subsection{Create a Release}
\label{createarelease}

\begin{enumerate}
\item Change directories into bosh-sample-release:

\begin{verbatim}
cd ~/bosh-sample-release
\end{verbatim}


This directory contains the sample WordPress deployment and release files. If this were a Cloud Foundry deploy, you would work with analogous files, provided by your BOSH contact.

\item Reset your environment

\begin{verbatim}
bosh reset release
\end{verbatim}


\item Answer \texttt{yes} to the prompt and wait for the environment to be reset

\item Create a release

\begin{verbatim}
bosh create release –force –with-tarball
\end{verbatim}


\item Answer \texttt{wordpress} to the \texttt{release name}prompt

\item Your terminal will display information about the release including the Release Manifest, Packages, Jobs, and tarball location.

\item Open \texttt{bosh-sample-release\slash wordpress.yml} in your favorite text editor and confirm that \texttt{name} is \texttt{wordpress} and \texttt{version} matches the version that was displayed in your terminal (if this is your first release, this will be version 1).

\end{enumerate}

\subsection{Deploy the Release}
\label{deploytherelease}

\begin{enumerate}
\item Upload the WordPress example Release to your Environment

\begin{verbatim}
bosh upload release dev_releases/wordpress-1.tgz
\end{verbatim}


\item Your terminal will display information about the upload, and an upload progress bar will reach 100\% after a few minutes.

\item Open \texttt{bosh-sample-release\slash wordpress.yml} and make sure that your networking and IP addresses match the environment that you were given. An example manifest is in the Appendix.

\item Deploy the Release

\begin{verbatim}
bosh deploy
\end{verbatim}


\item Your deployment will take a few minutes.

\item Copy the URL for your WordPress installation from the \texttt{properties.wordpress.servername} value in \texttt{wordpress.yml}

\item Browse your WordPress blog at this URL.

\item Complete the form to install your WordPress blog

\end{enumerate}

\begin{figure}[htbp]
\centering
\includegraphics[keepaspectratio,width=\textwidth,height=0.75\textheight]{deployed.pdf}
\caption{Deployed Wordpress}
\label{}
\end{figure}



\section{Appendix}
\label{appendix}

\begin{adjustwidth}{2.5em}{2.5em}
\begin{verbatim}

    ---
    name: wordpress
    director_uuid: 4a8a029c-f0ae-49a2-b016-c8f47aa1ac85

    release:
        name: wordpress
        version: 1

    compilation:
        workers: 4
        network: default
        cloud_properties:
            ram: 2048
            disk: 8096
            cpu: 2

    # this section describes how updates are handled

    update:
        canaries: 1
        canary_watch_time: 30000
        update_watch_time: 30000
        max_in_flight: 4
        max_errors: 1

    networks:
    - name: default
        subnets:
        - reserved:
            - 172.23.224.2 - 172.23.224.10
            - 172.23.224.200 - 172.23.224.254
        static:
            - 172.23.224.11 - 172.23.224.100
            range: 172.23.224.0/23
            gateway: 172.23.224.1
        dns:
            - 172.22.22.153
            - 172.22.22.154
        cloud_properties:
            name: VLAN2224

    - name: dmz
      subnets:
      - static:
        - 172.20.4.241 - 172.20.4.242
      range: 172.20.4.241/28
      dns:
        - 172.22.22.153
        - 172.22.22.154
      cloud_properties:
        name: VLAN3079

    resource_pools:

        - name: infrastructure
        network: default
        size: 6
        stemcell:
            name: bosh-stemcell
            version: 0.4.4
        cloud_properties:
            cpu: 1
            disk: 8192
            ram: 4096

    jobs:
        - name: mysql
        template: mysql
        instances: 1
        resource_pool: infrastructure
        persistent_disk: 16384
        networks:
            - name: default
            static_ips:
                - 172.23.224.11

        - name: wordpress
        template: wordpress
        instances: 4
        resource_pool: infrastructure
        networks:
            - name: default
            static_ips:
                - 172.23.224.12 - 172.23.224.15

        - name: nginx
        template: nginx
        instances: 1
        resource_pool: infrastructure
        networks:
            - name: default
                default: [dns, gateway]
                static_ips:
                    - 172.23.224.1
            - name: dmz
                static_ips:
                    - 172.20.4.241

    properties:
        wordpress:
            admin: foo@bar.com
            port: 8008
            servers:
                    - 172.23.224.12
                    - 172.23.224.13
                    - 172.23.224.14
                    - 172.23.224.15
            servername: wp.cf48.dev.las01.vcsops.com
            db:
                name: wp
                user: wordpress
                pass: w0rdpr3ss
            auth_key: random key
            secure_auth_key: random key
            logged_in_key: random key
            nonce_key: random key
            auth_salt: random key
            secure_auth_salt: random key
            logged_in_salt: random key
            nonce_salt: random key
        mysql:
            address: 172.23.224.11
            port: 3306
            password: verysecretpasswordforroot
        nginx:
            workers: 1

\end{verbatim}
\end{adjustwidth}

\input{mmd-memoir-footer}

\end{document}
